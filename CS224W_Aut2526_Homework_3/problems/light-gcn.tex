\section{LightGCN [13 points]}
\noindent\textit{Coverage: This problem is covered in Lecture 11.}

We learned in class about \textbf{LightGCN}, a GNN model for recommender systems. Given a bipartite user-item graph $G = (V, E)$, let $\mathbf{A} \in \mathbb{R}^{|V| \times |V|}$ be its unnormalized adjacency matrix, $\mathbf{D} \in \mathbb{R}^{|V| \times |V|}$ be its degree matrix and $\mathbf{E}^{(k)} \in \mathbb{R}^{|V| \times d}$ be its node embedding matrix at layer $k$ where $d$ is the embedding dimension. Let $\tilde{\mathbf{A}} = \mathbf{D}^{-1/2} \mathbf{A} \mathbf{D}^{-1/2}$ be the normalized adjacency matrix.\\

\noindent The original GCN updates node embeddings across layers according to $\mathbf{E}^{(k+1)} = \text{ReLU}(\tilde{\mathbf{A}} \mathbf{E}^{(k)} \mathbf{W}^{(k)})$, while LightGCN removes the non-linearity and uses the equation for each layer $k\in \{0, 1, ..., K-1\}$:
\begin{equation}
    \mathbf{E}^{(k+1)} = \tilde{\mathbf{A}} \mathbf{E}^{(k)}
\end{equation}
Moreover, LightGCN adopts multi-scale diffusion to compute the final node embeddings for link prediction, averaging across layers:
\begin{equation}\label{eq:lightgcn-diff}
    \mathbf{E} = \sum_{i=0}^{K} \alpha_{i} \mathbf{E}^{(i)},
\end{equation}
where we have uniform coefficients $\alpha_{i} = \frac{1}{K + 1}$.


\subsection{Advantages of Average Embeddings [4 points]}
    Why does LightGCN aerage over layer embeddings? What benefits does it bring, in a recommendation systems setting?

    \textbf{What to submit?} 1-3 sentences of explanation on the reasons and benefits of averaging across layers.

% your solution here
\Solution{}

\subsection{Self-connection [4 points]}
We denote the embedding of an item $i$ at layer-k $\mathbf{e}_i^{(k)}$ and that of a user $u$ $\mathbf{e}_u^{(k)}$. The graph convolution operation (a.k.a., propagation rule) in LightGCN is defined as:
$$\mathbf{e}^{(k+1)}_u = \sum_{i\in\mathcal{N}_u}\frac{1}{\sqrt{\lvert\mathcal{N}_u\rvert}\sqrt{\lvert\mathcal{N}_i\rvert}}\mathbf{e}^{(k)}_i$$
$$\mathbf{e}^{(k+1)}_i = \sum_{u\in\mathcal{N}_i}\frac{1}{\sqrt{\lvert\mathcal{N}_i\rvert}\sqrt{\lvert\mathcal{N}_u\rvert}}\mathbf{e}^{(k)}_u$$  
The symmetric normalization term$\frac{1}{\sqrt{\lvert\mathcal{N}_u\rvert}\sqrt{\lvert\mathcal{N}_i\rvert}}$follows the design of standard GCN, which can avoid the scale of embeddings increasing with graph convolution operations.\\ 

\noindent However, from the equations above, we can find that in LGCN, we only aggregate the connected neighbors and do not integrate the target node itself (i.e., there is no \textbf{self-connection}). This is different from most existing graph convolution operations that typically aggregate extended neighbors and also specifically handle self-connection.\\

\noindent Does LightGCN contain implicit self-connection? If your answer is yes, which operation captures the same effect as self-connection? If no, what do you think is the reason why LightGCN doesn't need self-connection or similar effects?

\textbf{What to submit?} Yes or no and 1-2 sentences of justification.

% your solution here
\Solution{}

\subsection{Relation with APPNP [5 points]}
There is a work that connects GCN with Personalized PageRank, where the authors propose a GCN variant named APPNP that can propagate long range without the risk of oversmoothing. Inspired by the teleport design in Personalized PageRank, APPNP complements each propagation layer with the starting features (i.e., the 0-th layer embeddings), which can balance the need of preserving locality (i.e., staying close to the root node to alleviate oversmoothing) and leveraging the information from a large neighborhood. The propagation layer in APPNP is defined as:
$$\mathbf{E}^{(k+1)} = \beta\mathbf{E}^{(0)}+(1-\beta)\tilde{\mathbf{A}}E^{(k)}$$
where $\beta$ is called the ``teleport probability'' to control the retention of starting features in the propagation, and $\tilde{\mathbf{A}}$ denotes the normalized adjacency matrix.\\

\noindent Aligning with Equation (\ref{eq:lightgcn-diff}), we can see that by setting $\alpha_k$ accordingly, LightGCN can fully recover the prediction embedding used by APPNP. As such, LightGCN shares the strength of APPNP in combating oversmoothing — by setting the $\alpha$ properly, LightGCN allows using a large K for long-range modeling with controllable oversmoothing.\\

\noindent Express the layer-$K$ embeddings $\mathbf{E}^{(K)}$ of APPNP as a function of the initial embeddings $\mathbf{E}^{(0)}$ and the normalized adjacency matrix $\tilde{\mathbf{A}}$. Show all work.

 \textbf{What to submit?} Multi-line mathematical derivation of the relationship between $\mathbf{E}^{(K)}$ and $\mathbf{E}^{(0)}$

% your solution here
 \Solution{}